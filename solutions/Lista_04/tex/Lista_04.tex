\documentclass{article}

\usepackage[utf8]{inputenc}
\usepackage[brazil]{babel}
\usepackage{amsmath}
\usepackage{amsfonts}
\usepackage[section]{placeins}
\usepackage{booktabs}
\usepackage{multirow}
\usepackage{listings}
\usepackage{booktabs}
\usepackage[title]{appendix}

\usepackage{caption}

\usepackage{hanging}
\usepackage{graphicx}

\usepackage{natbib}
\usepackage{float}

% Commands --------------------------------------------------------------------

\newcommand\indep{\protect\mathpalette{\protect\independenT}{\perp}}
\def\independenT#1#2{\mathrel{\rlap{$#1#2$}\mkern2mu{#1#2}}}

\setlength{\parskip}{0.5em}

\def\arraystretch{1.5}

% Document basics -------------------------------------------------------------

\title{Métodos Numéricos - Problem set 05}
\author{Samuel de S. Barbosa}
\date{June, 2018}

\begin{document}

\maketitle

In this problem set we will numerically solve a simple savings problem in a
economy with idiosyncratic shocks.

Suppose there is a continuum of goat farmers that are subject to endowment
shocks. A farmer's endowment is $e^z$, where $z$ follows the following
stochastic process:

$$z^\prime = \rho z + \epsilon,$$

where $\epsilon \sim N(0,\sigma^2)$. The farmers instantaneous utility
function is given by

$$u(c) = \frac{c^{1-\gamma}-1}{1-\gamma}$$

and he discounts the future with the factor $\beta \in (0,1)$. Each farmer has
access to a storage technology such that, if he sets aside q goats today, he
will have 1 goat tomorrow. His budget constraint can then be writte as:

$$c + qa^\prime = e^z + a$$

Let $\beta = q = 0.96$ and $\gamma = 1.0001$ for now.

\section*{1.a)}

Let $\rho = 0.9$ and $\sigma = 0.01$. Using the Tauchen method to discretize the
stochastic process in a Markov chain with 9 states, with 3 standard deviations
for each side, we have the following grid for $e^z$ and transition matrix

\begin{scriptsize}
\begin{tabular}{lllllllll}
   $e^z$ = [0.9335 &   0.9497  &  0.9662  &  0.9829  &  1.0000   & 1.0174 &   1.0350 &   1.0530  &  1.0712]
\end{tabular}

P = \begin{bmatrix}
    0.5683 & 0.4025 & 0.0290 & 0.0002 & 0.0000 & 0.0000 &      0 &      0 &      0 \\[0.3em]
    0.0843 & 0.5503 & 0.3459 & 0.0194 & 0.0001 & 0.0000 & 0.0000 &      0 &      0 \\[0.3em]
    0.0017 & 0.1125 & 0.5829 & 0.2902 & 0.0126 & 0.0000 & 0.0000 & 0.0000 &      0 \\[0.3em]
    0.0000 & 0.0029 & 0.1480 & 0.6034 & 0.2376 & 0.0080 & 0.0000 & 0.0000 & 0.0000 \\[0.3em]
    0.0000 & 0.0000 & 0.0049 & 0.1899 & 0.6104 & 0.1899 & 0.0049 & 0.0000 & 0.0000 \\[0.3em]
    0.0000 & 0.0000 & 0.0000 & 0.0080 & 0.2376 & 0.6034 & 0.1480 & 0.0029 & 0.0000 \\[0.3em]
    0.0000 & 0.0000 & 0.0000 & 0.0000 & 0.0126 & 0.2902 & 0.5829 & 0.1125 & 0.0017 \\[0.3em]
    0.0000 & 0.0000 & 0.0000 & 0.0000 & 0.0001 & 0.0194 & 0.3459 & 0.5503 & 0.0843 \\[0.3em]
    0.0000 & 0.0000 & 0.0000 & 0.0000 & 0.0000 & 0.0002 & 0.0290 & 0.4025 & 0.5683 \\[0.3em]
\end{bmatrix}
\end{scriptsize}

\section*{1.b)}

Now we discretize the asset space using a grid starting from the natural debt limit under
the worst endowment state up to two times the savings under the best state. 
This gives us a grid in $[-23.3373, 53.5624]$, with 1.000 points.

Solving the individual goat farmer problem for each state variable, using value
function iteration on Matlab, we get the following value and policy functions:

\begin{figure}[H]
  \centering
    \includegraphics[width=1.1\textwidth]{graf1.png}
\end{figure}

\section*{1.c)}

Next we find the stationary distribution $\pi(z, a)$ and use it to compute the aggregate savings in the
economy. We then compute the aggregate savings from

	$$ A = \sum_a \sum_z \pi(a,z) a^\prime(a,z) \,\, , $$

which gives us an aggregate savings of 26.49.

\section*{1.d)}

Now we redo the analysis using $\rho = 0.97$. Farmers now are exposed to more
variance over endowments, and longer streaks of the same type of endowment
states:

\begin{scriptsize}
\begin{tabular}{lllllllll}
   $e^z$ = [0.8839  &  0.9116 &   0.9402&    0.9696 &   1.0000 &   1.0313 &   1.0636 &   1.0970  &  1.1313]
\end{tabular}

P = \begin{bmatrix}
    0.8795  &  0.1205  &  0.0000  &  0.0000  &       0  &       0  &       0  &       0  &       0 \\[0.3em]
    0.0344  &  0.8627  &  0.1029  &  0.0000  &  0.0000  &       0  &       0  &       0  &       0 \\[0.3em]
    0.0000  &  0.0420  &  0.8707  &  0.0873  &  0.0000  &  0.0000  &       0  &       0  &       0 \\[0.3em]
    0.0000  &  0.0000  &  0.0510  &  0.8755  &  0.0735  &  0.0000  &  0.0000  &       0  &       0 \\[0.3em]
    0.0000  &  0.0000  &  0.0000  &  0.0615  &  0.8771  &  0.0615  &  0.0000  &  0.0000  &       0 \\[0.3em]
    0.0000  &  0.0000  &  0.0000  &  0.0000  &  0.0735  &  0.8755  &  0.0510  &  0.0000  &  0.0000 \\[0.3em]
    0.0000  &  0.0000  &  0.0000  &  0.0000  &  0.0000  &  0.0873  &  0.8707  &  0.0420  &  0.0000 \\[0.3em]
    0.0000  &  0.0000  &  0.0000  &  0.0000  &  0.0000  &  0.0000  &  0.1029  &  0.8627  &  0.0344 \\[0.3em]
    0.0000  &  0.0000  &  0.0000  &  0.0000  &  0.0000  &  0.0000  &  0.0000  &  0.1205  &  0.8795 \\[0.3em]
\end{bmatrix}
\end{scriptsize}

Thus, being in low endowment states are now worse than before, and, on the
other side, being on high states are better. This gives us value functions
that are more spread over states.

To avoid high disutility from low asset states, agents now save a little bit
more on every state (policy function slightly shifted to the right), resulting
in an aggregate savings of 27.15.

\section*{1.e)}

If we suppose $\gamma = 5$, agents are now much more risk averse than
before. This means farmers want a more smooth consumption path than before,
so savings go up a lot under better states. Value functions are more concave, and 
aggregate savings is 48.67.

\section*{1.f)}

If we suppose $\sigma=0.05$, variance over endowment states gets even larger
than we saw on (d):

\begin{scriptsize}
\begin{tabular}{lllllllll}
   $e^z$ = [0.7088  &  0.7725  &  0.8419  &  0.9176 &  1.0000  &  1.0898  &  1.1878 &   1.2945  &  1.4108]
\end{tabular}
\end{scriptsize}

Transitions over states does not change relatively to (c), so
besides higher variance over endowment, there is more uncertainty going on.
Thus, as expected, aggregate savings (34.11) is higher than we found on (c) and on (d).

\begin{figure}[p]
    \vspace*{-2cm}
    \makebox[\linewidth]{
        \includegraphics[width=1.3\linewidth]{graf5.png}
    }
\end{figure}

\end{document}
